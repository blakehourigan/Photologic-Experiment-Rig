\documentclass{article}

\usepackage{hyperref}
\usepackage{listings}
\usepackage{xcolor}
\usepackage{graphicx}
\usepackage{subcaption}
\usepackage{placeins}
\usepackage{pdfpages}

\graphicspath{{"/home/blake/Pictures/arduino migration guide"}}

\setlength{\parskip}{1em}

\title{Samuelsen Lab\\CNC Machine User Guide}
\author{Blake Hourigan}
\date{April 30, 2025}

\lstset{
    basicstyle=\ttfamily\small,
    breaklines=true,
    numbers=left,
    numberstyle=\tiny,
    frame=single,
    keywordstyle=\color{blue},
    commentstyle=\color{gray},
    stringstyle=\color{orange},
}
\lstdefinelanguage{yaml}{
    morekeywords={true,false,null,y,n},      % Add keywords here
    sensitive=false,                        % Case insensitive
    morecomment=[l]{\#},                    % Line comments start with #
    morestring=[b]",                        % Double-quoted strings
    morestring=[b]',                        % Single-quoted strings
}

\begin{document}
\maketitle
\clearpage
\tableofcontents
\clearpage


\section{Sourcing a Part File (Finding the File to Cut a Shape)}
\subsection{The End Goal}
What you will eventually need to obtain to run a cutting operation on the Shapeoko Pro CNC machine is a \textit{C2D} file. You can identify these files
by their file \textit{extension}. What this means is that CNC machine files will look something like \textit{`part\_name\_here.c2d'}.

Some parts, especially those which have already been cut out before using the Shapeoko Pro CNC machine will already have a c2d file (also referred to 
in this document as a \textit{toolpath} file). If this is the case and you have located the C2D file for your part you can move on to Section 
\ref{sec:verifying-toolpath}, where the process of verifying and modifying existing toolpaths is detailed.

For newly created parts, a C2D file will not exist. You will need to obtain or create a \textit{part file} containting the dimensions of the part 
that you would like to cut out. Then, from this part file, you will need to generate a \textit{`dxf'} file. This `dxf' file will contain the 
two-dimensional representation of some `face' of a part, which will allow you to create a `toolpath' that will cut this `face' out from your stock. 

So, to briefly summarize, to cut a part out from a piece of stock on the Shapeoko Pro CNC Machine, you need a C2D file. This file may or may not already
exist. If it does exist and you have located the file, you can modify this for your particular piece of stock and cut it out. If it does not exist you
will need to create it. To create a C2D you need:
\begin{itemize}
    \item A part file (Solidworks SLDPRT or Fusion360 STEP). This can be created by you or sourced from online. But you need to be reasonable about 
        what you can cut out on our machine. The simpler the better.
    \item A \textit{dxf} file. This is a 2d representation of some side or `face' of the part that can be cut out by the CNC. 
\end{itemize}

Once you have obtained the dxf file, you can place it in a tool called \textit{Carbide Create} to define the toolpath. This is discussed further 
in Section \ref{sec:create-toolpath}.

\subsection{What is a Part File?}
A part file is a file used for 3d modelling purposes. There are many types of files such as \textit{SLDPRT} files, \textit{STEP} files, \textit{STL} files
and more. These are all similar in that they represent a 3d object in digital form. Most importantly for our use case, they contain information about 
objects including the \textit{dimensions} of the part we would like to cut out in the real world. 

The type of file that should interest us for the purposes of creating a C2D file are SLDPRT files, which are files created for parts inside of the 
Solidworks 3d modelling program, OR STEP files, which can be used in Solidworks or Autodesk Fusion360.

\subsection{Where do Part Files Come From?}

\subsection{Where do Part Files Come From?}


\section{Creating a Toolpath (Making a C2D File)}
\label{sec:create-toolpath}

\subsection{What is a Toolpath?}

\subsection{What Tools will I Use?}

\subsection{Verifying/Modifying Toolpaths}
\label{sec:verifying-toolpath}

\section{Securing the Part to the Cutting Board}
This part of the process is mostly independent of the material that you will be cutting, but one must be mindful of the material througout the 
process. This will be discussed in more detail when relevent below.

\section{Securing the Part Using Tape}

\section{Securing the Part Using Glue}
This method is generally most useful when cutting aluminum, but can be very useful with acryllic as well. When possible, it is more efficient to avoid 
using the glue method because one must clean the cutting board thoroughly each time a new stock piece (piece of material that you cut a part out of) is 
placed on the cutting board. Furthermore, one must glue the piece to the cutting board and wait for the glue to \textit{cure} which takes about 30
minutes until the part is secure enough to stay put through an entire tool path cycle.

For most acryllic tool paths, the tape method described previously works fantastically well and can be much quicker and more efficient than using the 
glue method with all its overhead. If a tool path is very complex, using much or most of the area of the stock piece, it may be useful to use the glue
method. When using aluminum, the glue method can be incredibly useful to keep the entire sheet as level as possible, which is important to achieve 
clean cuts on all cut edges.

\end{document}
